\documentstyle[11pt]{article}
\begin{document}
\begin{center} {\bf
WC(1L)
} \end{center}
\begin{description}


\item[NAME] \hfill \\
       wc - print the number of bytes, words, and lines in files

\item[SYNOPSIS] \hfill \\
       wc   [-clw]   [-\hspace{.01cm}-bytes]   [-\hspace{.01cm}-chars]   [-\hspace{.01cm}-lines]  [-\hspace{.01cm}-words]
       [file...]

\item[DESCRIPTION] \hfill \\
       This manual page documents the  GNU  version  of  wc.   wc
       counts  the  number  of bytes, whitespace-separated words,
       and newlines in each given file, or the standard input  if
       none  are  given  or  when  a file named `-' is given.  It
       prints one line of counts for each file, and if  the  file
       was given as an argument, it prints the filename following
       the counts.  If more than one filename is given, wc prints
       a  final  line  containing the cumulative counts, with the
       filename `total'.  The counts are printed  in  the  order:
       lines, words, bytes.

       By default, wc prints all three counts.  Options can specify 
       that only certain counts be printed.  Options  do  not
       undo others previously given, so wc -\hspace{.01cm}-bytes -\hspace{.01cm}-words prints
       both the byte counts and the word counts.

\item[OPTIONS] \hfill \\
       -c, -\hspace{.01cm}-bytes, -\hspace{.01cm}-chars \\
              Print only the byte counts.

       -w, -\hspace{.01cm}-words \\
              Print only the word counts.

       -l, -\hspace{.01cm}-lines \\
              Print only the newline counts.

       The long-named options can be introduced with `+' as  well
       as  `-\hspace{.01cm}-', for compatibility with previous releases.  Eventually 
       support for `+' will  be  removed,  because  it  is incompatible 
       with the POSIX.2 standard.
\end{description}
\end{document}
