\documentstyle[11pt]{article}
\begin{document}
\begin{center} {\bf
PR(1L)
} \end{center}
\begin{description}


\item[NAME] \hfill \\
       pr - convert text files for printing

\item[SYNOPSIS] \hfill \\
       pr [+PAGE] [-COLUMN] [-abcdfFmrtv] [-e[in-tab-char[in-tab-
       width]]] [-h header] [-i[out-tab-char[out-tab-width]]] [-l
       page-length]   [-n[number-separator[digits]]]   [-o  left-
       margin] [-s[column-separator]] [-w page-width] [file...]

\item[DESCRIPTION] \hfill \\
       This manual page documents the  GNU  version  of  pr.   pr
       prints  on  the standard output a paginated and optionally
       multicolumn copy of the text files given  on  the  command
       line,  or  of  the standard input if no files are given or
       when the file name `-' is encountered.  Form feeds in  the
       input cause page breaks in the output.

\item[OPTIONS] \hfill \\
       +PAGE   \\
              Begin printing with page PAGE.

       -COLUMN \\
              Produce  COLUMN-column  output  and  print  columns
              down.  The column width is automatically  decreased
              as  COLUMN  increases; unless you use the -w option
              to increase the page width  as  well,  this  option
              might cause some columns to be truncated.

       -a      \\
              Print columns across rather than down.

       -b       \\
              Balance columns on the last page.

       -c      \\
              Print  control characters using hat notation (e.g.,
              `\^G'); print other unprintable characters in  octal
              backslash notation.

       -d      \\
              Double space the output.

       -e[in-tab-char[in-tab-width]] \\
              Expand  tabs to spaces on input.  Optional argument
              in-tab-char is the  input  tab  character,  default
              tab.   Optional  argument in-tab-width is the input
              tab character's width, default 8.

       -F, -f  \\
              Use a formfeed instead of newlines to separate output pages.

       -h header \\
              Replace  the filename in the header with the string
              header.

       -i[out-tab-char[out-tab-width]] \\
              Replace spaces with tabs on output.  Optional argu-
              ment  out-tab-char  is  the  output  tab character,
              default tab.  Optional  argument  out-tab-width  is
              the output tab character's width, default 8.

       -l page-length \\
              Set  the  page  length  to  page-length lines.  The
              default is 66.  If page-length is less than 10, the
              headers  and  footers  are  omitted,  as  if the -t
              option had been given.

       -m      \\
              Print all files in parallel, one in each column.

       -n[number-separator[digits]] \\
              Precede each column with a line number; with paral-
              lel  files,  precede  each line with a line number.
              Optional argument number-separator is the character
              to  print after each number, default tab.  Optional
              argument digits is the number of  digits  per  line
              number, default 5.

       -o left-margin \\
              Offset  each  line with a margin left-margin spaces
              wide.  The total page width is this offset plus the
              width set with the -w option.

       -r      \\
              Do  not  print  a  warning message when an argument
              file cannot be opened.   Failure  to  open  a  file
              still makes the exit status nonzero, however.

       -s[column-separator] \\
              Separate  columns  by  the single character column-
              separator, default tab, instead of spaces.

       -t      \\
              Do not print  the  5-line  header  and  the  5-line
              trailer  that are normally on each page, and do not
              fill out the bottoms of pages (with blank lines  or
              formfeeds).

       -v      \\
              Print  unprintable  characters  in  octal backslash notation.

       -w page-width \\
              Set the page  width  to  page-width  columns.   The
              default is 72.
\end{description}
\end{document}
