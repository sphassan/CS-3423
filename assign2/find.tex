\documentstyle[11pt]{article}
\begin{document}
\begin{center} {\bf
FIND(1L)
} \end{center}
\begin{description}


\item[NAME] \hfill \\
       find - search for files in a directory hierarchy

\item[SYNOPSIS] \hfill \\
       find [path...] [expression]

\item[DESCRIPTION] \hfill \\
       This  manual page documents the GNU version of find.  find
       searches the directory tree rooted at each given  pathname
       by  evaluating  the  given  expression from left to right,
       according to the rules of precedence (see  section  OPERA-
       TORS),  until  the outcome is known (the left hand side is
       false for and operations, true for  or),  at  which  point
       find moves on to the next pathname.

       The first argument that begins with `-', `(', `)', `,', or
       `!' is taken to be the beginning of  the  expression;  any
       arguments before it are paths to search, and any arguments
       after it are the rest of the expression.  If no paths  are
       given, the current directory is used.  If no expression is
       given, the expression `-print' is used.

       find exits with status 0 if all files are  processed  suc-
       cessfully, greater than 0 if errors occur.

\item[EXPRESSIONS] \hfill \\
       The expression is made up of options (which affect overall
       operation rather than the processing of a  specific  file,
       and  always  return  true),  tests (which return a true or
       false value), and actions (which  have  side  effects  and
       return a true or false value), all separated by operators.
       -and is assumed where the operator  is  omitted.   If  the \\
       expression  contains  no actions other than -prune, -print
       is performed on all files  for  which  the  expression  is
       true.

\item[OPTIONS] \hfill \\
       All options always return true.

       -daystart \\
              Measure  times  (for  -amin, -atime, -cmin, -ctime,
              -mmin, and -mtime)  from  the  beginning  of  today \\
              rather than from 24 hours ago.

       -depth  \\
              Process each directory's contents before the directory itself.

       -follow \\
              Dereference symbolic links.  Implies -noleaf.

       -maxdepth levels \\
              Descend at most  levels  (a  non-negative  integer)
              levels  of directories below the command line argu-
              ments.  `-maxdepth 0' means only  apply  the  tests
              and actions to the command line arguments.

       -mindepth levels \\
              Do  not  apply  any tests or actions at levels less
              than levels (a non-negative  integer).   `-mindepth
              1'  means process all files except the command line
              arguments.

       -noleaf \\
              Do not optimize by assuming that  directories  con-
              tain  2  fewer  subdirectories than their hard link
              count.  Each directory on a normal Unix  filesystem
              has  at  least  2  hard links: its name and its `.'
              entry.  Additionally, its subdirectories  (if  any)
              each  have  a `..'  entry linked to that directory.
              When find is examining a directory,  after  it  has
              statted 2 fewer subdirectories than the directory's
              link count, it knows that the rest of  the  entries
              in  the directory are non-directories (`leaf' files
              in the directory tree).  If only the  files'  names
              need to be examined, there is no need to stat them;
              this gives a significant increase in search  speed.
              However, this option might be needed when searching
              NFS-mounted non-Unix file systems, if they  do  not
              follow   the  directory-link  convention  described
              above.

       -version \\
              Print find version number on standard error.

       -xdev   \\
              Don't descend directories on other filesystems.

\item[TESTS] \hfill \\
       Numeric arguments can be specified as

       +n     for greater than n, \\

       -n     for less than n, \\

       n      for exactly n.

       -amin n \\
              File was last accessed n minutes ago.

       -anewer file \\
              File was last accessed more recently than file  was
              modified.   -anewer  is affected by -follow only if
              -follow comes before -anewer on the command line. \\

       -atime n \\
              File was last accessed n*24 hours ago.

       -cmin n \\
              File's status was last changed n minutes ago.

       -cnewer file \\
              File's status was last changed more  recently  than
              file  was modified.  -cnewer is affected by -follow
              only if -follow comes before -cnewer on the command
              line.

       -ctime n \\
              File's status was last changed n*24 hours ago.

       -empty  \\
              File  is  empty  and  is either a regular file or a
              directory.

       -false  \\
              Always false.

       -fstype type \\
              File is on a filesystem of type  type.   The  valid
              filesystem  types  vary among different versions of
              Unix; an incomplete list of filesystem  types  that
              are accepted on some version of Unix or another is:
              ufs, 4.2, 4.3, nfs, tmp, mfs, S51K, S52K.

       -gid n  \\
              File's numeric group ID is n.

       -group gname \\
              File belongs  to  group  gname  (numeric  group  ID
              allowed).

       -inum n \\
              File has inode number n.

       -links n \\
              File has n links.

       -lname pattern \\
              File  is a symbolic link whose contents match shell
              pattern pattern.  The metacharacters do  not  treat
              `/' or `.' specially.

       -mmin n \\
              File's data was last modified n minutes ago.

       -mtime n \\
              File's data was last modified n*24 hours ago.

       -name pattern \\
              Base of path name (the path with the leading direc-
              tories removed) matches shell pattern pattern.  The
              metacharacters  (`*', `?', and `[]') do not match a
              `.' at the start of the base name.

       -newer file \\
              File was modified more recently than file.   -newer
              is affected by -follow only if -follow comes before
              -newer on the command line. \\

       -nouser \\
              No user corresponds to file's numeric user ID.

       -nogroup \\
              No group corresponds to file's numeric group ID.

       -path pattern \\
              Path  name  matches  shell  pattern  pattern.   The
              metacharacters do not treat `/' or `.' specially.

       -perm mode \\
              File's  permission  bits are exactly mode (octal or
              symbolic).  Symbolic modes use mode 0 as a point of
              departure.

       -perm -mode \\
              All  of  the  permission  bits mode are set for the
              file.

       -perm +mode \\
              Any of the permission bits mode  are  set  for  the
              file.

       -regex pattern \\
              Pathname matches regular expression pattern.

       -size n[ck] \\
              File  uses  n 512-byte blocks (bytes if `c' follows
              n, kilobytes if `k' follows n).  The size does  not
              count  indirect  blocks,  and  does count blocks in
              sparse files that are not actually allocated.

       -true   \\
              Always true.

       -type c \\
              File is of type c:

              b      block (buffered) special

              c      character (unbuffered) special

              d      directory

              p      named pipe (FIFO)

              f      regular file

              l      symbolic link

              s      socket

       -uid n  \\
              File's numeric user ID is n.

       -used n \\
              File was last accessed n days after its status  was
              last changed.

       -user uname \\
              File  is  owned  by  user  uname  (numeric  user ID
              allowed).

       -xtype c \\
              The same as -type unless the  file  is  a  symbolic
              link.   For symbolic links, if -follow has not been
              given, true if the file is a link to a file of type
              c;  if  -follow  has  been given, true if c is `l'.
              For symbolic links, -xtype checks the type  of  the
              file that -type does not check.

\item[ACTIONS] \hfill \\
       -exec command ; \\
              Execute command; true if 0 status is returned.  All
              following arguments to find are taken to  be  arguments
              to  the command until an argument consisting
              of `;' is encountered.  The string `{}' is replaced
              by  the current pathname being processed everywhere
              it occurs in the arguments to the command, not just
              in arguments where it is alone, as in some versions
              of find.  Both of these constructions might need to
              be  escaped  (with a `\verb+\+') or quoted to protect them
              from expansion by the shell.

       -fprint file \\
              True; print the full pathname into file  file.   If
              file  does  not  exist when find is run, it is cre-
              ated; if it does exist, it is truncated.  The file-
              names  ``/dev/stdout'' and ``/dev/stderr'' are han-
              dled specially; they refer to the  standard  output
              and standard error output, respectively.

       -fprint0 file \\
              True;  like -print0 but write to file like -fprint.

       -fprintf file format \\
              True; like -printf but write to file like  -fprint.

       -ok command ; \\
              Like  -exec but ask the user first (on the standard
              input); if the response does not start with `y'  or
              `Y', do not run the command, and return false.

       -print True;   \\
              print the full pathname on the standard out-
              put, followed by a newline.

       -print0 \\
              True; print the full pathname on the standard  out-
              put,  followed  by  a  null character.  This allows
              filenames that contain  newlines  to  be  correctly
              interpreted  by programs that process the find out-
              put.

       -printf format \\
              True; print format on the standard  output,  inter-
              preting  `\verb+\+'  escapes  and  `\%'  directives.  Field
              widths and precisions can be specified as with  the
              `printf'  C  function.  Unlike -print, -printf does
              not add a newline at the end of  the  string.   The
              escapes and directives are:

              \verb+\+a     Alarm bell.

              \verb+\+b     Backspace.

              \verb+\+c     Stop  printing from this format immediately.

              \verb+\+f     Form feed.

              \verb+\+n     Newline.

              \verb+\+r     Carriage return.

              \verb+\+t     Horizontal tab.

              \verb+\+v     Vertical tab.

              \verb+\+\verb+\+     A literal backslash (`\verb+\+').

              A `\verb+\+' character followed by any other character  is
              treated  as an ordinary character, so they both are
              printed.

              \%\%     A literal percent sign.

              \%a     File's  last  access  time  in  the   format
                     returned by the C `ctime' function.

              \%Ak    File's last access time in the format speci-
                     fied by k, which is either `@' or  a  direc-
                     tive  for  the  C  `strftime' function.  The
                     possible values for k are listed below; some
                     of  them  might not be available on all sys-
                     tems,  due  to  differences  in   `strftime'
                     between systems.

                      @      seconds  since  Jan.  1, 1970, 00:00
                             GMT.

                     Time fields:

                      H      hour (00..23)

                      I      hour (01..12)

                      k      hour ( 0..23)

                      l      hour ( 1..12)

                      M      minute (00..59)

                      p      locale's AM or PM

                      r      time, 12-hour (hh:mm:ss [AP]M)

                      S      second (00..61)

                      T      time, 24-hour (hh:mm:ss)

                      X      locale's time representation (H:M:S)

                      Z      time zone (e.g., EDT), or nothing if
                             no time zone is determinable

                     Date fields:

                      a      locale's  abbreviated  weekday  name
                             (Sun..Sat)

                      A      locale's full weekday name, variable
                             length (Sunday..Saturday)

                      b      locale's  abbreviated   month   name
                             (Jan..Dec)

                      B      locale's  full  month name, variable
                             length (January..December)

                      c      locale's date and time (Sat  Nov  04
                             12:02:33 EST 1989)

                      d      day of month (01..31)

                      D      date (mm/dd/yy)

                      h      same as b

                      j      day of year (001..366)

                      m      month (01..12)

                      U      week  number  of year with Sunday as
                             first day of week (00..53)

                      w      day of week (0..6)

                      W      week number of year with  Monday  as
                             first day of week (00..53)

                      x      locale's     date     representation
                             (mm/dd/yy)

                      y      last two digits of year (00..99)

                      Y      year (1970...)

              \%b     File's size in 512-byte blocks (rounded up).

              \%c     File's last status change time in the format
                     returned by the C `ctime' function.

              \%Ck    File's last status change time in the format
                     specified by k, which is the same as for \%A.

              \%d     File's depth in the directory tree; 0  means
                     the file is a command line argument.

              \%f     File's pathname with any leading directories
                     removed.

              \%g     File's group name, or numeric  group  ID  if
                     the group has no name.

              \%G     File's numeric group ID.

              \%h     Leading directories of file's pathname.

              \%H     Command  line  argument under which file was
                     found.

              \%i     File's inode number (in decimal).

              \%k     File's size in 1K blocks (rounded up).

              \%l     Object of symbolic  link  (empty  string  if
                     file is not a symbolic link).

              \%m     File's permission bits (in octal).

              \%n     Number of hard links to file.

              \%p     File's pathname.

              \%P     File's pathname with the name of the command
                     line  argument  under  which  it  was  found
                     removed.

              \%s     File's size in bytes.

              \%t     File's  last modification time in the format
                     returned by the C `ctime' function.

              \%Tk    File's last modification time in the  format
                     specified by k, which is the same as for \%A.

              \%u     File's user name, or numeric user ID if  the
                     user has no name.

              \%U     File's numeric user ID.

              A  `\%' character followed by any other character is
              discarded (but the other character is printed).

       -prune If -depth is not given, true; do  not  descend  the \\
              current directory.
              If -depth is given, false; no effect.

       -ls    True;  list  current  file  in `ls -dils' format on \\
              standard  output.   The  block  counts  are  of  1K
              blocks,    unless    the    environment    variable
              POSIXLY_CORRECT is  set,  in  which  case  512-byte
              blocks are used.

\item[OPERATORS] \hfill \\
       Listed in order of decreasing precedence:

       ( expr )
              Force precedence.

       ! expr True if expr is false.

       -not expr \\
              Same as ! expr.

       expr1 expr2
              And  (implied);  expr2 is not evaluated if expr1 is
              false.

       expr1 -a expr2
              Same as expr1 expr2.

       expr1 -and expr2
              Same as expr1 expr2.

       expr1 -o expr2
              Or; expr2 is not evaluated if expr1 is true.

       expr1 -or expr2
              Same as expr1 -o expr2.

       expr1 , expr2
              List; both expr1 and expr2  are  always  evaluated.
              The  value  of expr1 is discarded; the value of the
              list is the value of expr2.
\end{description}
\end{document}
