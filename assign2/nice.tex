\documentstyle[11pt]{article}
\begin{document}
\begin{center} {\bf
NICE(1L)
} \end{center}
\begin{description}


\item[NAME] \hfill \\
       nice - run a program with modified scheduling priority

\item[SYNOPSIS] \hfill \\
       nice     [-n    adjustment]    [-adjustment]    [-\hspace{.01cm}-adjustment=adjustment] [command [arg...]]

\item[DESCRIPTION] \hfill \\
       This manual page documents the GNU version of nice.  If no
       arguments  are  given,  nice  prints the current sheduling
       priority, which it inherited.  Otherwise,  nice  runs  the
       given  command  with its scheduling priority adjusted.  If
       no adjustment is given, the priority  of  the  command  is
       incremented  by  10.  The superuser can specify a negative
       adjustment.  The priority can be adjusted by nice over the
       range of -20 (the highest priority) to 19 (the lowest).

\item[OPTIONS] \hfill \\
       -n adjustment, -adjustment, -\hspace{.01cm}-adjustment=adjustment \\
              Add  adjustment instead of 10 to the command's priority.

       The long-named options can be introduced with `+' as  well
       as  `-\hspace{.01cm}-', for compatibility with previous releases.  Eventually 
       support for `+' will  be  removed,  because  it  is
       incompatible with the POSIX.2 standard.
\end{description}
\end{document}
