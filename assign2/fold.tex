\documentstyle[11pt]{article}
\begin{document}
\begin{center} {\bf
FOLD(1L)
} \end{center}
\begin{description}


\item[NAME] \hfill \\
       fold - wrap each input line to fit in specified width

\item[SYNOPSIS] \hfill \\
       fold [-bs] [-w width] [-\hspace{.01cm}-bytes] [-\hspace{.01cm}-spaces] [-\hspace{.01cm}-width=width]
       [file...]

\item[DESCRIPTION] \hfill \\
       This manual page documents the GNU version of fold.   fold
       prints  the specified files, or the standard input when no
       files are given or the filename `-' is encountered, on the
       standard  output.   It  breaks  long  lines  into multiple
       shorter lines by inserting a newline  at  column  80.   It
       counts screen columns, so tab characters usually take more
       than one column, backspace characters decrease the  column
       count, and carriage return characters set the column count
       back to zero.

\item[OPTIONS] \hfill \\
       -b, -\hspace{.01cm}-bytes \\
              Count bytes rather  than  columns,  so  that  tabs,
              backspaces,  and  carriage returns are each counted
              as taking up one column, just  like  other  charac-
              ters.

       -s, -\hspace{.01cm}-spaces \\
              Break at word boundaries.  If the line contains any
              blanks, the line is broken  after  the  last  blank
              that  falls  within  the  maximum  line length.  If
              there are no blanks, the line is broken at the max-
              imum line length, as usual.

       -w, -\hspace{.01cm}-width width \\
              Use  a maximum line length of width columns instead
              of 80.

       The long-named options can be introduced with `+' as  well
       as  `-\hspace{.01cm}-', for compatibility with previous releases.  Even-
       tually support for `+' will  be  removed,  because  it  is
       incompatible with the POSIX.2 standard.
\end{description}
\end{document}
